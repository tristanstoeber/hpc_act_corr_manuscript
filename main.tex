\documentclass{article}
%% Language and font encodings
\usepackage[english]{babel}
\usepackage[utf8x]{inputenc}
\usepackage[T1]{fontenc}
\usepackage{tcolorbox}
\usepackage{listings}
%% Sets page size and margins
\usepackage[a4paper,top=3cm,bottom=2cm,left=3cm,right=3cm,marginparwidth=1.75cm]{geometry}

%% Useful packages
\usepackage{amsmath}
\usepackage{graphicx}
\usepackage[colorlinks=true, allcolors=black]{hyperref}
%\usepackage{subfig}
%\usepackage{cite}
\usepackage{natbib}
\usepackage{multirow}
\usepackage{diagbox}
\usepackage{bbold}
\usepackage[font=small]{caption}
%\usepackage[labelformat = empty,position=top]{subcaption}
\usepackage[position=top]{subcaption}
\usepackage[export]{adjustbox}
\usepackage[inline]{enumitem}
\usepackage{comment}
\usepackage{svg}
\usepackage{fullpage}
\usepackage{times}
\usepackage[]{ccaption}
\usepackage{fancyhdr,graphicx,amsmath,amssymb}
\usepackage[ruled,vlined]{algorithm2e}
\usepackage{breqn}
\usepackage{authblk}
\usepackage{placeins}
\usepackage{cleveref}

\include{pythonlisting}
% figure float
\usepackage{float}

\usepackage{enumitem}
\usepackage{bbding}
\usepackage{breqn}

\setlength{\parskip}{1em}
%\setlength{\parindent}{0pt}
\newcommand\wordcount{
    \immediate\write18{texcount -sub=section main_framework.tex  | grep "Section" | sed -e 's/+.*//' | sed -n \thesection p > 'count.txt'}
(\input{count.txt}words)}
\newcommand\lword[1]{\leavevmode\nobreak\hskip0pt plus\linewidth\penalty50\hskip0pt plus-\linewidth\nobreak\textbf{#1}}
\newcommand{\beginsupplement}{%
        \setcounter{table}{0}
        \renewcommand{\thetable}{S\arabic{table}}%
        \setcounter{figure}{0}
        \renewcommand{\thefigure}{S\arabic{figure}}%
     }

% Necessary for dashed multicolumn
%\newcommand{\fulldashfill}{\hspace{0.1em}\leaders\hbox{-\hspace{0.1em}}\hfill\kern0pt\hspace{0.1em}}
%\newcommand{\dashmulticolumn}[3]{\multicolumn{#1}{#2}{\fulldashfill #3 \fulldashfill}}
\newcommand{\fulldashfill}{\leaders\hbox{-\hspace{0.05em}}\hfill\kern0pt}
\newcommand{\dashmulticolumn}[3]{\multicolumn{#1}{#2}{\fulldashfill{}#3\fulldashfill}}

% Line numbers
\usepackage{lineno}
\linenumbers
%newcommand{\R}[1]{\label{#1}\linelabel{#1}}
\newcommand{\R}[1]{\linelabel{#1}}
%\newcommand{\lr}[1]{page~\pageref{#1}, line~\lineref{#1}}
\newcommand{\lr}[1]{\lineref{#1}}

\usepackage{setspace}
\doublespacing

\title{Activity correlations across the hippocampal formation}
\author[1]{Co-authors} % to be replaced later
\author[2,3,\Envelope]{Tristan M. St\"ober}
\affil[1]{Students IDS group 37}
\affil[2]{Institute for Neural Computation, Ruhr University Bochum, Germany}
\affil[3]{Department of Neurology, University Hospital Frankfurt, Frankfurt, Germany}
\affil[\Envelope]{Correspondence: tristan.stoeber@posteo.net}

\begin{document}

\maketitle

\section{Introduction}
\begin{itemize}
    \item There is an unresolved conundrum: Plasticity between CA3 and CA2 is strongly limited by a number of mechanisms \citep{zhao2007synaptic, lee2010rgs14, carstens2016perineuronal}, however generally unlocking this plasticity improves learning performance in a spatial navigation task\citep{lee2010rgs14}. So if it improves learning, why is it limited in the first place \citep{lehr2021ca2}?
    \item In \cite{stober2020selective} we postulated that the limited plasticity between CA3-CA2 allows to prioritize specific salient experiences during subsequent replay by allowing co-active cell assemblies in CA3 and CA2 to interact upon the release of neuromodulation.
    \item Such interactions should lead to differences in the correlation structure of neurons in CA3 and CA2: Cells co-active during an important experience should increase their coordinated activity shortly after the experience and this should last for some time. In contrast, a similar experience without a saliency signal should not be accompanied by such an increase.
    \end{itemize}

\section{Methods}
\subsection{Data}
\begin{itemize}
    \item Large scale recordings with high density probes of CA3 and CA2 pyramidal neurons have been conducted and are available upon request \citep{oliva2020hippocampal}. These data include recordins with and wihtout salient experiences - empty area vs social stimulus.
    \item Access to the data has already been requested. However, until then, we can develop the full analysis pipeline on freely available online data \citep{mizuseki2014neurosharing}\footnote{https://crcns.org/data-sets/hc/hc-3/about-hc-3}.
    \item Such data is already spike sorted. Thus we will have the time points of action potentials for each units. Units are defined as clusters of spikes for a given tetrode.
    \item TODO: Tristan - Explore colab for storing and analysing data
\end{itemize} 

\subsection{Pair-wise correlations}
Determining the correlation between individual pairs of neurons is straight forward: 
\begin{enumerate}
    \item Activity is binned - counting the number of action potentials for each neuron per time window.
    \item Vectors of neural activity can be correlated with preimplented Python functions such as numpy.correlate
\end{enumerate} 

\subsection{Computing platform}
To allow everyone access to computing power and storage, its easiest if we work together on Juypter notebooks hosted on Google Colab.

\section{Results}
\subsection{Baseline correlations}
Characterize correlation structure. Key questions here: What is the baseline correleation between different neurons within and across hippocampal subregions? How does it change with time?

Can we identify clusters of correlated neurons? 

\begin{figure}[H]
 \centering
  \begin{subfigure}[b]{0.29\textwidth}
     \centering
     \includesvg[height=0.17\textheight]{figures/fig_1a_corr_suggstd.svg}
     \caption{}
     \label{fig:SeqComp_ActivationFunction}
 \end{subfigure}
 \hfill     
    \caption{
    \textbf{Baseline correlations across hippocampal subregions.}
    \textbf{a) Suggested Figure: Example correlation matrix across hippocampal subregions for a given during behavior XXX.  }
}
    \label{fig:SeqComp_SingleSequence}
\end{figure}


\section{Milestones}
\begin{enumerate}
    \item Load freely available data and convert activity levels of individual neurons into vectors
    \item 
    \item Add CA3-CA2 recordings and study the effect of social experiences on the correlation structure
\end{enumerate}
\bibliographystyle{plainnat}
\bibliography{bib}

\end{document}